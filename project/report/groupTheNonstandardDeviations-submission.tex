\documentclass[10pt,conference,compsocconf]{IEEEtran}
\usepackage{booktabs}
\usepackage[T1]{fontenc}
\usepackage[backend=biber,style=ieee]{biblatex}
\bibliography{groupTheNonstandardDeviations-literature}
\usepackage{graphicx}
\usepackage{hyperref}


\begin{document}
\title{Computational Intelligence Laboratory Project: Collaborative Filtering}

\author{\IEEEauthorblockN{%
  Dina Zverinski\IEEEauthorrefmark{1},
  Jan Wilken D\"orrie\IEEEauthorrefmark{2} and
  \'Alvaro Marco A\~n\'o\IEEEauthorrefmark{3}}

\IEEEauthorblockA{%
  Group: TheNonstandardDeviations\\
  Department of Computer Science, ETH Zurich, Switzerland\\
  Email: \IEEEauthorrefmark{1}zdina@student.ethz.ch,
  \IEEEauthorrefmark{2}dojan@student.ethz.ch,
  \IEEEauthorrefmark{3}malvaro@student.ethz.ch}
}
\maketitle

\begin{abstract}
  % Short description of the whole paper, to help the reader decide whether to
  % read it.

  % The abstract should really be written last, along with the title of the
  % paper. The four points that should be covered:
  %   - State the problem.
  %   - Say why it is an interesting problem.
  %   - Say what your solution achieves.
  %   - Say what follows from your solution.

  Lorem ipsum dolor sit amet, consetetur sadipscing elitr, sed diam nonumy
  eirmod tempor invidunt ut labore et dolore magna aliquyam erat, sed diam
  voluptua. At vero eos et accusam et justo duo dolores et ea rebum. Stet clita
  kasd gubergren, no sea takimata sanctus est Lorem ipsum dolor sit
  amet.\cite{funk2006netflix}
\end{abstract}

\section{Introduction}
\label{sec:introduction}
% Describe your problem and state your contributions.

Lorem ipsum dolor sit amet, consetetur sadipscing elitr, sed diam nonumy eirmod
tempor invidunt ut labore et dolore magna aliquyam erat, sed diam voluptua. At
vero eos et accusam et justo duo dolores et ea rebum. Stet clita kasd gubergren,
no sea takimata sanctus est Lorem ipsum dolor sit amet.

\section{Models and Methods}
\label{sec:models_and_methods}
% Describe your idea and how it was implemented to solve the problem. Survey the
% related work, giving credit where credit is due.

% The models and methods section should describe what was done to answer the
% research question, describe how it was done, justify the experimental design,
% and explain how the results were analyzed.

% The model refers to the underlying mathematical model or structure which you
% use to describe your problem, or that your solution is based on.  The methods
% on the other hand, are the algorithms used to solve the problem.  In some
% cases, the suggested method directly solves the problem, without having it
% stated in terms of an underlying model. Generally though it is a better
% practice to have the model figured out and stated clearly, rather than
% presenting a method without specifying the model. In this case, the method can
% be more easily evaluated in the task of fitting the given data to the
% underlying model.

% The methods part of this section, is not a step-by-step, directive, protocol
% as you might see in your lab manual, but detailed enough such that an
% interested reader can reproduce your work.

% The methods section of a research paper provides the information by which a
% study's validity is judged.  Therefore, it requires a clear and precise
% description of how an experiment was done, and the rationale for why specific
% experimental procedures were chosen.  It is usually helpful to structure the
% methods section by:
% - Layout the model you used to describe the problem or the solution.
% - Describing the algorithms used in the study, briefly including details such
%   as hyperparameter values (e.g. thresholds), and preprocessing steps (e.g.
%   normalizing the data to have mean value of zero).
% - Explaining how the materials were prepared, for example the images used and
%   their resolution.
% - Describing the research protocol, for example which examples were used for
%   estimating the parameters (training) and which were used for computing
%   performance.
% - Explaining how measurements were made and what calculations were performed.
%   Do not reproduce the full source code in the paper, but explain the key
%   steps.

Lorem ipsum dolor sit amet, consetetur sadipscing elitr, sed diam nonumy eirmod
tempor invidunt ut labore et dolore magna aliquyam erat, sed diam voluptua. At
vero eos et accusam et justo duo dolores et ea rebum. Stet clita kasd gubergren,
no sea takimata sanctus est Lorem ipsum dolor sit amet.


\section{Results}
\label{sec:results}
% Show evidence to support your claims made in the introduction.

% Organize the results section based on the sequence of table and figures you
% include. Prepare the tables and figures as soon as all the data are analyzed
% and arrange them in the sequence that best presents your findings in a logical
% way. A good strategy is to note, on a draft of each table or figure, the one
% or two key results you want to address in the text portion of the results.
% The information from the figures is summarized in Table.

% When reporting computational or measurement results, always report the mean
% (average value) along with a measure of variability (standard deviation(s) or
% standard error of the mean).

Lorem ipsum dolor sit amet, consetetur sadipscing elitr, sed diam nonumy eirmod
tempor invidunt ut labore et dolore magna aliquyam erat, sed diam voluptua. At
vero eos et accusam et justo duo dolores et ea rebum. Stet clita kasd gubergren,
no sea takimata sanctus est Lorem ipsum dolor sit amet.


\section{Discussion}
\label{sec:discussion}
% Discuss the strengths and weaknesses of your approach, based on the results.
% Point out the implications of your novel idea on the application concerned.

Lorem ipsum dolor sit amet, consetetur sadipscing elitr, sed diam nonumy eirmod
tempor invidunt ut labore et dolore magna aliquyam erat, sed diam voluptua. At
vero eos et accusam et justo duo dolores et ea rebum. Stet clita kasd gubergren,
no sea takimata sanctus est Lorem ipsum dolor sit amet.


\section{Summary}
\label{sec:summary}
% Summarize your contributions in light of the new results.

Lorem ipsum dolor sit amet, consetetur sadipscing elitr, sed diam nonumy eirmod
tempor invidunt ut labore et dolore magna aliquyam erat, sed diam voluptua. At
vero eos et accusam et justo duo dolores et ea rebum. Stet clita kasd gubergren,
no sea takimata sanctus est Lorem ipsum dolor sit amet.

\printbibliography%
\end{document}
