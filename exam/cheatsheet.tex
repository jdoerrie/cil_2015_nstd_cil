\documentclass[conference,11pt]{IEEEtran}
\usepackage{amsmath}
\usepackage{amssymb}
\usepackage{amsthm}
\usepackage[margin=1cm]{geometry}
\usepackage{bm}
\newtheorem*{thm}{Theorem}

\DeclareMathOperator*{\argmin}{\arg\!\min}
\DeclareMathOperator*{\argmax}{\arg\!\max}

\newcommand{\iver}[1]{\left[#1\right]}
\newcommand{\card}[1]{\left\lvert#1\right\rvert}
\newcommand{\abs}[1]{\left\lvert#1\right\rvert}
\newcommand{\norm}[1]{\left\lVert#1\right\rVert}
\newcommand{\matr}[1]{\boldsymbol{\mathbf{#1}}}
\newcommand{\vect}[1]{\boldsymbol{\mathbf{#1}}}
\newcommand{\trns}[1]{#1^{\top}}
\newcommand{\normal}{\mathcal{N}}
\newcommand{\R}{\mathbb{R}}

\begin{document}
\title{CIL Cheat Sheet 2015}
\author{Jan Wilken D\"orrie}
\maketitle

\section{Linear Algebra Primer}
\subsection{Equivalent Conditions}
For $\matr{A} \in \R^{M \times M}$ the following conditions
are equivalent:
\begin{itemize}
  \item $\matr{A}$ has an inverse $\matr{A}^{-1}$,
  \item $\mathrm{rank}(\matr{A}) = M$,
  \item $\mathrm{range}(\matr{A}) = \R^M$,
  \item $\mathrm{null}(\matr{A}) = \{\vect{0}\}$,
  \item $0$ is not an eigenvalue of $\matr{A}$,
  \item $0$ is not a singular value of $\matr{A}$
\end{itemize}


\section{Norms}
\subsection{Vector norms}
A \emph{norm} is a function $\norm{\bullet} : V \to \R$ quantifying the
size of a vector. It must satisfy
\begin{enumerate}
  \item Positive scalability: $\norm{a \cdot \vect{x}} = \abs{a} \cdot
    \norm{\vect{x}}$ for $a \in \R$
  \item Triangle inequality: $\norm{\vect{x} + \vect{y}} \leq
    \norm{\vect{x}} + \norm{\vect{y}}$ for $\vect{x}, \vect{y} \in V$.
  \item Seperability: $\norm{\vect{x}} = 0$ implies $\vect{x} = 0$.
\end{enumerate}

\begin{itemize}
  \item The most commonly used norms are the \emph{$p$-norms}:
    \[
      \norm{\vect{x}}_p := {\left(\sum_{i=1}^n \abs{x_i}^p \right)}^{1/p}
    \]
  \item A special case is the \emph{Euclidean norm}
    \[ \norm{\vect{x}}_2 := \sqrt{\sum_{i=1}^n x_i^2} \]
  \item The ``\emph{$0$-norm}'' (not really a norm) is
    \( \norm{\vect{x}}_0 := \card{\{x_i \mid x_i \neq 0\}} \)
\end{itemize}

\subsection{Matrix norms}
We can also define norms on matrices, satisfying the properties described
above. $\matr{A} \in \R^{M \times N}$:
\begin{itemize}
  \item \emph{Frobenius norm}:
    \[
      \norm{\matr{A}}_F := \sqrt{\sum_{i=1}^M\sum_{j=1}^N a_{ij}^2}
      = \sqrt{\sum_{i=1}^{K} \sigma_i^2}, \quad K = \min(M,N)
    \]
    Only depends on singular values of $\matr{A}$

  \item \emph{$p$-norms for matrices}:
    \(
      \norm{\matr{A}}_p := \sup\{\norm{\matr{A}\vect{x}}_p :
      \norm{\vect{x}}_p = 1\}
    \)
  \item \emph{Euclidean or spectral norm}:
    \[
      \norm{\matr{A}}_2 := \sup\{\norm{\matr{A}\vect{x}}_2 :
      \norm{\vect{x}}_2 = 1\} = \sigma_1,
    \]
    the largest singular value.

\end{itemize}

\section{Dimension Reduction}
\subsection{Principal Component Analysis (PCA)}
Orthogonal linear projection of high dimensional data onto low
dimensional subspace. Objectives:
\begin{enumerate}
  \item Minimize error $\norm{\vect{x} - \tilde{\vect{x}}}_2$ of
    point $\vect{x}$ and its approximation $\tilde{\vect{x}}$.
  \item Preserve information: maximize variance.
\end{enumerate}
Both objectives are shown to be formally equivalent.
\subsubsection{Statistics of Projected Data}
\begin{itemize}
  \item Mean of the data: sample mean $\bar{\vect{x}}$
  \item Covariance of the data:
    \[
      \matr{\Sigma} = \frac{1}{N} \sum_{n=1}^N (\vect{x}_n -
      \bar{\vect{x}}) \trns{(\vect{x}_n - \bar{\vect{x}})}
    \]
\end{itemize}
\subsubsection{Solution: Eigenvalue Decomposition}
The eigenvalue decomposition of the covariance matrix
$\matr{\Sigma} = \matr{U}\matr{\Lambda}\trns{\matr{U}}$ contains
all relevant information.
\begin{itemize}
  \item For $K \leq D$ dimensional projection space: Choose $K$
    eigenvectors $\{\vect{u}_1, \ldots, \vect{u}_K\}$ with largest
    associated eigenvalues $\{\lambda_1, \ldots, \lambda_K\}$.
\end{itemize}

\subsection{Singular Value Decomposition}
\begin{thm}[Eckart-Young]
  Let $\matr{A}$ be a matrix of rank $R$, if we wish to
  approximate $\matr{A}$ using a matrix of a lower rank $K$
  then, $\tilde{\matr{A}} ̃= \sum_{k=1}^K d_k \vect{u}_k
  \trns{\vect{v}_k}$ is the closest matrix in the Frobenius norm.
  (Assumes ordering of singular values $d_k \geq d_{k+1}$)
\end{thm}

\section{Clustering}
\subsection{K-Means}
\subsubsection{Motivation}
\begin{itemize}
  \item Given: set of data points $\vect{x}_1, \ldots, \vect{x}_N \in \R^D$
  \item Goal: find \emph{meaningful partition} of the data
    \begin{itemize}
      \item i.e.\ a labeling of each data point with a unique label
        \[
          \pi: \{1,\ldots,N\} \to \{1,\ldots,K\} \text{ or }
          \pi: \R^D \to \{1,\ldots,K\}
        \]
      \item note: numbering of clusters is arbitrary
      \item $k$-th cluster recovered by $\pi^{-1}(k) \subseteq \{1,\ldots,N\}$
        or $\subseteq \R^D$
    \end{itemize}
\end{itemize}

\subsubsection{Vector Quantization}
\begin{itemize}
  \item Partition of the space $\R^D$
  \item Clusters represented by \emph{centroids} $\vect{u}_k \in \R^D$
  \item Mapping induced via nearest centroid rule
    \[ \pi(\vect{x}) = \argmin_{k=1,\ldots,K} \norm{\vect{u}_k - \vect{x}}_2 \]
\end{itemize}

\subsubsection{Objective Function for $K$-Means}
\begin{itemize}
  \item Useful notation: represent $\pi$ via indicator matrix $\matr{Z}$:
    \[ z_{kn} := \iver{\pi(\vect{x}_n) = k} \]
  \item $K$-means Objective function
    \[
      J(\matr{U},\matr{Z}) = \sum_{n=1}^N \sum_{k=1}^K z_{kn} \norm{\vect{x}_n
        - \vect{u}_k}_2^2 = \norm{\matr{X} - \matr{U}\matr{Z}}_F^2,
    \]
    where $\matr{X} = [\vect{x}_1,\ldots,\vect{x}_N] \in \R^{D \times N}$
    and   $\matr{U} = [\vect{u}_1,\ldots,\vect{u}_K] \in \R^{D \times K}$
\end{itemize}

\subsubsection{$K$-means Algorithm: Optimal Assignment}
\begin{itemize}
  \item Compute optimal assignment $\matr{Z}$, given centroids $\matr{U}$
    \begin{itemize}
      % \item each data point contributes to exactly one term in outer sum
      \item minimize each column of $\matr{Z}$ separately
        \[
          \vect{z}_{\bullet n}^* = \argmin_{z_{1n},\ldots,z_{Kn}} \sum_{k=1}^K
          z_{kn} \norm{\vect{x}_n - \vect{u}_k}_2^2
        \]
      \item optimum is attained by mapping to the closest centroid
        \[
          z_{kn}^*(\matr{U}) = \iver{k = \argmin_l \norm{\vect{x} - \vect{u}_l}_2}
        \]
    \end{itemize}
\end{itemize}

\subsubsection{$K$-means Algorithm: Optimal Assignment}
\begin{itemize}
  \item Compute optimal choice of $\matr{U}$, given assignments $\matr{Z}$
    \begin{itemize}
      \item continuous variables: compute gradient and set to zero (necessary
        optimality condition)
      \item look at (partial) gradient for every centroid $\vect{u}_k$
        \[
          \nabla_{\vect{u}_k} J(\matr{U},\matr{Z}) = \sum_{n=1}^N z_{kn}
          \nabla_{\vect{u}_k} \norm{\vect{x}_n - \vect{u}_k}_2^2 = -2
          \sum_{n=1}^N z_{kn} (\vect{x}_n - \vect{u}_k)
        \]
      \item setting gradient to zero
        \[
          \nabla_{\matr{U}} J(\matr{U},\matr{Z}) \stackrel{!}{=} 0 \implies
          \vect{u}_k^*(\matr{Z}) = \frac{\sum_{n=1}^N z_{kn} \vect{x}_n}
          {\sum_{n=1}^N z_{kn}} %\text{, if } \sum_{n=1}^N z_{kn} > 0
        \]
    \end{itemize}
\end{itemize}

\subsubsection{$K$-means Algorithm: Analysis}
\begin{itemize}
  \item Computational cost of each iteration is $O(KND)$
  \item $K$-means convergence is guaranteed
  \item $K$-means optimizes a non-convex objective. Hence we are not guaranteed
    to find the global optimum.
  \item Finds a local optimum $(\matr{U},\matr{Z})$ in the following sense
    \begin{itemize}
      \item for each $\matr{Z}'$ with $\frac{1}{2}\norm{\matr{Z}-\matr{Z}'}_0 = 1$
        (differs in one assignment)
      \item $J(\matr{U}^*(\matr{Z}'), \matr{Z}') \geq J(\matr{U},\matr{Z})$
      \item may gain by changing assignments of $\geq 2$ points
    \end{itemize}
  \item $K$-means algorithm can be used to compress data
    \begin{itemize}
      \item with information loss, if $K < N$
      \item store only the centroids and the assignments
    \end{itemize}
\end{itemize}
% Alternate between two steps until convergence:
% \begin{enumerate}
%   \item Update assignments $z_{k,n}$ of data points to centroids:
%     \[
%       z_{k,n} = \begin{cases}
%         1, &\text{if $k = \argmin_j \norm{\vect{x}_n - \vect{u}_j}_2$} \\
%         0, &\text{else}
%       \end{cases}
%     \]
%   \item Update centroid positions:
%     \[
%       \vect{u}_k = \frac{\sum_n z_{k,n} \vect{x}_n}{\sum_n z_{k,n}}
%     \]
% \end{enumerate}

\subsection{Mixture Models}
\subsubsection{Gaussian Mixture Model (GMM)}
\[
  p_{\theta}(\vect{x}) = \sum_{k=1}^K \pi_k \normal(\vect{x} \mid \vect{\mu}_k,
  \matr{\Sigma}_k),
\]
where $\pi_k \geq 0$, $\sum_{k=1}^K \pi_k = 1$.

\subsubsection{Complete Data Distribution}
\begin{itemize}
  \item Explicitly introduce latent variables in the generative model
  \item Assignment variable (for a generic data point) $z_k \in \{0,1\}$,
    $\sum_{k=1}^K z_k = 1$
  \item We have that $\Pr(z_k = 1) = \pi_k$ or $p(\vect{z}) = \prod_{k=1}^K
    \pi_k^{z_k}$
  \item Joint distribution over $(\vect{x}, \vect{z})$ (\emph{complete data}
    distribution) $p(\vect{x}, \vect{z}) = \prod_{k=1}^K {\left[ \pi_k
        \normal(\vect{x} \mid \vect{\mu}_k, \matr{\Sigma}_k)\right]}^{z_k}$
\end{itemize}

\subsubsection{Posterior Assignments}
\emph{Posterior probabilities} for assignments
\[
  \Pr(z_k = 1 \mid \vect{x}) = \frac{\pi_k \normal(\vect{x} \mid \vect{\mu}_k,
    \matr{\Sigma}_k)}{\sum_{l=1}^K \pi_l \normal(\vect{x} \mid \vect{\mu}_l,
    \matr{\Sigma}_l)}
\]

\subsubsection{Lower Bounding the Log-Likelihood}
\begin{itemize}
  \item Expectation Maximization
    \begin{itemize}
      \item maximize a lower bound on the log-likelihood
      \item systematic way of deriving a family of bounds
      \item based on complete data distribution
    \end{itemize}
  \item Specifically:
    \begin{align*}
      \ln p_{\theta}(\vect{x}) &= \ln \sum_{\vect{z}} p_{\theta}(\vect{x},
      \vect{z}) = \ln \sum_{k=1}^K p(\vect{x}, \theta_k) \pi_k \\
      &= \ln \sum_{k=1}^K q_k \frac{p(\vect{x}; \theta_k) \pi_k}{q_k} \\
      &\geq \sum_{k=1}^K q_k \left[\ln p(\vect{x},\theta_k)+\ln\pi_k-\ln q_k \right]
    \end{align*}
    \begin{itemize}
      \item follows from Jensen's inequality (concavity of logarithm)
      \item can be done for the contribution of each data point (additive)
    \end{itemize}
\end{itemize}

\subsubsection{Mixture Model: Expectation Step}
\[
  q_k = \frac{\pi_k\; p(\vect{x}; \theta_k)}{\sum_{l=1}^K \pi_l\;p(\vect{x}, \theta_l)}
  = \Pr(z_k = 1 \mid \vect{x})
\]

\subsubsection{Mixture Model: Maximization Step}
\[ \pi_k^* = \frac{1}{N} \sum_{n=1}^N q_{kn} \]
\[ \vect{\mu}_k^* = \frac{\sum_{n=1}^N q_{kn}\vect{x}_n}{\sum_{n=1}^N q_{kn}} \]
\[
  \matr{\Sigma}_k^* = \frac{\sum_{n=1}^N
    q_{kn}\vect{x}_n\trns{\vect{x}_n}}{\sum_{n=1}^N q_{kn}}
\]

\subsubsection{AIC and BIC}
\begin{itemize}
  \item Trade-off: achieve balance between data fit --- measured by likelihood
    $p(\matr{X}\mid\theta)$ --- and complexity. Complexity can be measured by
    the number of free parameters $\kappa(\cdot)$.
  \item Different Heuristics for choosing $K$
    \begin{itemize}
      \item \emph{Akaike Information Criterion} (AIC)
        \[
          \mathrm{AIC}(\theta\mid\matr{X}) = -\ln p_{\theta}(\matr{X}) +
          \kappa(\theta)
        \]
      \item \emph{Bayesian Information Criterion} (BIC)
        \[
          \mathrm{BIC}(\theta\mid\matr{X}) = -\ln p_{\theta}(\matr{X}) +
          \frac{1}{2} \kappa(\theta) \ln N
        \]
    \end{itemize}
  \item Generally speaking, the BIC criterion penalizes complexity more than
    the AIC criterion.
\end{itemize}

\section{Sparse Coding}

\section{Robust PCA}

\end{document}
